% Creating a simple Title Page in Beamer
\documentclass{beamer}

% Theme choice:
\usetheme{AnnArbor}
% \usetheme{CambridgeUS}
% \usetheme{Antibes}

% Title page details: 
\title{Your First \LaTeX{} Presentation}
% \subtitle{My-subtitle}
\author{latex-beamer.com}
\institute{Online Beamer Tutorials}

% Multiple authors
\author{First~Author \and 
  Second~Author \and
  Third~Author \and
  Fourth~Author \and
   Fifth~Author}
\begin{document}

% Title page frame
\begin{frame}
    \titlepage
\end{frame}

 % Outline frame
\begin{frame}{Outline}
  \tableofcontents
 \end{frame}

% % % Presentation structure
\section{Problem statement}
 \section{Existing results}
   \subsection{Method 1}
   \subsection{Method 2}
   \subsection{Method 3}
\section{Comparative study}
 \section*{References}

 \begin{frame}
 to enforce entries in the table of contents
\end{frame}

 Abstract environment
\begin{abstract}
  content
\end{abstract}

 \begin{frame}{Ordered Lists in Beamer}
\begin{enumerate}
    \item Item 1
    \item Item 2
    \item Item 3
\end{enumerate}
 \end{frame}

\begin{frame}{Lists in multiple frames}{Frame 1}
\begin{enumerate}
    \item Item 1
    \item Item 2
   \item Item 3
% % Store the actual item number
   \setcounter{currentenumi}{\theenumi}
\end{enumerate}
 \end{frame}

 \begin{frame}{Block environment}{Madrid theme}
 \begin{block}{Block title}
    It can be useful to treat some content differently by putting it into a block. This can be done by using blocks!
\end{block}
 \end{frame}

 \begin{frame}{Basic Blocks}
    \begin{block}{Standard Block}
        This is a standard block.
    \end{block}
    \begin{alertblock}{Alert Message}
        This block presents alert message.
    \end{alertblock}
    \begin{exampleblock}{An example of typesetting tool}
        Example: MS Word, \LaTeX{}
    \end{exampleblock}
\end{frame}


\begin{frame}{My first table}
 \begin{tabular}{|c||l||r|}
 \hline
\hline
    A & C & E\\ 
\hline
   B & D & F\\ 
\hline
\end{tabular}
\end{frame}


\end{document}
